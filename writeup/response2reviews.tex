\documentclass[11pt]{letter}
\usepackage[hidelinks]{hyperref}
\usepackage[green]{msuletterhead}
\usepackage[usenames,dvipsnames]{xcolor}
\pagenumbering{gobble}

\newcommand{\gb}[1]{{\bf\color{black}{#1}}}
\newcommand{\plr}[1]{{\it\color{purple}{(#1)}}}
\newcommand{\gc}[1]{{\it\color{blue}{(#1)}}}

\begin{document}

Re-Submission Cover Letter \hfill	\today\\
\emph{PLoS Genetics}
\\\\\\
To the Editor(s),

Please find enclosed the re-submission of our manuscript, 
``Inferring Continuous and Discrete Population Genetic Structure Across Space,"
for re-consideration for publication in PLoS Genetics.

We appreciate the opportunity to submit a revised version of the manuscript.  
The comments provided by the Guest Editor 
and two anonymous reviewers are very helpful, 
and we believe the manuscript is substantially improved as a result 
of their critiques.

Our detailed responses to those comments are inset below (in bold), 
but, in brief, WE HAVE DONE SOME STUFF.

We have also edited the text for 
greater clarity.\\\\
%
We hope that the paper is now acceptable for publication.\\\\
%
Thank you for your consideration,\\
Gideon Bradburd (for all of the authors).

\newpage

\textbf{Guest Editor}\\
We believe the concerns of reviewer 1 are addressable, and would like to see some effort put towards addressing them, specifically benchmarking against non-spatial structure identification methods.

\textbf{Reviewer 1}\\
The paper introduces a model-based method called conStruct to account for isolation-by-distance (IBD) when clustering individuals into populations. IBD is known to generate spurious clusters and the proposed software package can avoid spurious cluster detection. Accounting for space in the proposed model based framework amounts at introducing a regularization penalty when fitting admixture model. Regularization explains why overfitting is avoided and correct admixture proportions are found even when performing model-based clustering with a value of K that is too large.

This being said, I have serious concerns about the submitted manuscript. 

\begin{itemize}
\item First, many results concern comparisons between clustering based on the spatial and non-spatial models of construct (Figs 2, 3, 4, 6, 7, 9). However the authors should rather compare their spatial approach to existing software for model-based clustering, which do not account for space (Structure, fastSTRUCTURE, ADMIXTURE,...). Results obtained with one of these software should not be available in supplementary material but should rather be visible in the main text.\\\\
\gb{}\\

\item Second, the simulation scenario should correspond to a real biological scenario. Here this layer-based simulation scenario is very difficult to understand. I would be more interesting to stick to more classic spatially explicit simulations such as range expansion, or secondary contact scenarios.\\\\
\gb{}\\

\item Third, I think that the main point of the paper should be more explicit. Non-spatial model based approaches are prone to overfitting when K is too large and this can be avoided when using spatial-based approaches. \\\\
\gb{We disagree that this is the main point of the paper.}\\

\item Last, there are other software that account for space when performing model-based clustering and they should be included in the comparisons (BAPS, Geneland, TESS,...). Right now, they are only mentioned in the introduction and we would like to see results comparing these different model-based approaches.\\\\
\gb{
The methods mentioned here --- BAPS, Geneland, and TESS --- 
place spatial priors on admixture proportions
and/or cluster membership, 
but like STRUCTURE, they assume that allele frequencies 
within a cluster are constant across space.
Therefore, although spatially explicit, this parameterization does not 
account for space in a way that avoids the problem we describe in the paper: 
the characterization of spatially continuous differentiation 
as clines of admixture across spurious clusters.
STRUCTURE is the canonical model-based clustering algorithm
(having been cited $>$ 20,000 times), but is too slow to be run 
on the modern-scale SNP datasets we simulate, 
so fastSTRUCTURE, which implements an extremely similar model 
within a faster inference framework, 
provides the best comparison to our method.}\\

\item In the introduction, the authors write ``placing spatial priors on cluster memberships'' when referring to other methods that deal with ancestry proportions instead of cluster memberships.\\\\
\gb{We have edited our wording for clarity.}\\

\item Formula (1) about the covariance is difficult to understand. It does not correspond to a classic definition of a covariance. Does this correspond to the covariance of some random variables? The discussion following the definition of the covariance looks like a discussion and should be moved to the discussion.\\\\
\gb{
    The definition is a covariance of some random variables; we have made this fact more clear.
    We have also moved some of this discussion to the Discussion (heading ``Allelic or genetic covariance?'').
}\\

\end{itemize}

\textbf{Reviewer 2}\\
In this paper, Bradburd et al. present a novel analysis method for the joint characterization of discrete and continuous structure. Their model uses a mixed membership Gaussian process, adding a within layer (cluster) Isolation-by-distance-process. I do think that this paper is very clearly written, and presents the model and method in great detail. Besides the presentation of the method, a simulation and two application studies are performed, that lead to a comprehensive overview of the method. I would consider this model to be very valuable for a large audience in ecology and evolution, including biogeographers and conservation biologists, with the only drawback being the very high computational requirements. Overall, however I recommend the paper for acceptance with only very minor changes.

\begin{itemize}
\item One issue I am wondering about is whether there are some issues with ``outlier individuals'', that e.g. represent poorly sampled clusters, which might be an issue with the Alaskan/NW-Canadian bears. In particular, consider the case where the two layers were only fitted from the samples from the 48-states. Presumably, the resulting covariance curve would be similar to Fig S26b (assuming the two layer case). If we then added those NW-pacific samples, we might expect them to be most likely added to the cluster with the flatter right tail, even if they were essentially unrelated. This issue can also arise in kriging (Figure 6 in the first version of the preprint of Caye et al (https://www.biorxiv.org/content/biorxiv/early/2016/10/12/080291.full.pdf is the example I am thinking of). Could this interpretation be excluded somehow?
\\\\
\gb{}\\

\item Another thing that could be added are some additional simulations in more complex demographic models, such as expansions, temporally/spatially heterogeneous migration, or sparser sampling schemes. However, as the paper is already rather long (at almost 20 single-spaced pages), and the space of scenarios for which the model is relevant is enormous, I hope the authors consider further exploring their model in a follow-up study.\\\\
\gb{}\\

\item 100: Is there a reason that allelic covariance is used instead of genetic covariance?\\\\
\gb{
    Good question; we've added a paragraph on this to the Discussion.
}\\

\item Could supp. figures be sorted according to first mentioning in ms?\\\\
\gb{}\\

\item Fig3: increase label size\\\\
\gb{}\\

\item Last line in paragraph on p. 18: as the derivations are otherwise very detailed and easy to follow, it might be worth adding here that this follows from E[UV] = 4 Cov[A, B]\\\\
\gb{}\\

\item 634: Using ms,
\\\\
\gb{Fixed.}\\

\item Fig S4: Could the colors be made to correspond each other (e.g. panel e vs 3). In general, is the exchangeability of layers a practical concern, or similar to the original structure, does the MCMC never ``switch'' two layer labels?\\\\
\gb{}\\

\item Fig S26: maybe keep Y-axis scale consistent between panels?\\\\
\gb{}\\

\item Cross-Validation: As we all know SNP are not uncorrelated. I am wondering if that could influence the CV, as spurious patterns in the testing data might be partially explained by nearby SNP in the training data. Is that a concern?\\\\
\gb{}\\

\end{itemize}

\end{document}
