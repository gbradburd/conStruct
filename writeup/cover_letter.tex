\documentclass[11pt]{letter}
\usepackage[hidelinks]{hyperref}
\usepackage[green]{msuletterhead}
\begin{document}

Submission Cover Letter \hfill	\today\\
\emph{PLoS Genetics}
\\\\\\
To the Editor(s),

Please find enclosed our manuscript,  
``Inferring Continuous and Discrete Population Genetic Structure Across Space,"
for consideration for publication in PLoS Genetics.

In the manuscript, we present a novel statistical method 
for the simultaneous inference of continuous and discrete patterns of population structure.
The method estimates ancestry proportions for each 
sample from a set of discrete population layers, 
and, within each layer, estimates a rate at which relatedness decays with distance.
This model explicitly addresses the ``clines vs. clusters" problem in 
modeling population genetic variation by jointly accommodating 
continuous and discrete patterns of differentiation. 
We demonstrate the utility of this approach using simulations 
and two empirical applications.
We also release an associated R package, called \texttt{conStruct}, 
for ``\emph{con}tinuous \emph{struct}ure."

This work is motivated by the failure of existing model-based clustering methods, 
such as \texttt{STRUCTURE} (Pritchard \emph{et al.} 2000)
or \texttt{ADMIXTURE} (Alexander \emph{et al.} 2009), 
to model continuous patterns of geographical variaion (i.e., isolation by distance, or IBD).
Such geographical patterns are ubiquitous in nature,
but are described by these clustering methods using a possibly large number of discrete clusters.
These clusters are spurious, and can throw a 
wrench in the gears of efforts to describe and categorize natural variation, 
one of the first steps in any analysis of genetic data, 
and a principal mission of biology.  

Besides presenting a solution to this problem,
we use simulations and empirical data to demonstrate 
the ways in which nonspatial clustering approaches 
fail in the presence of IBD; 
and provide two different metrics for choosing an appropriate 
number of layers to describe the data.

As of this submission, \texttt{STRUCTURE} alone has been cited more than 20,000 times.
However, as the costs of sequencing continue to fall
and good geospatial information becomes more readily available,
the limitations we cite above will become more severe.
Our method provides an important extension to \texttt{STRUCTURE} and similar approaches, 
one that solves a critical and widely known basic conceptual drawback of the model. 
We therefore believe that our work represents an 
important and timely contribution to the field.

John Novembre (jnovembre@uchicago.edu), \\
Desislava Petkova (desislavka@gmail.com), \\
Jonathan Pritchard (pritch@stanford.edu),  \\
Sohini Ramachandran (sramachandran@brown.edu), \\
and Matthew Stephens (mstephens@uchicago.edu) \\
would be suitable reviewers. 

Sincerely,
Gideon Bradburd, 
Graham Coop, and 
Peter Ralph
\end{document}