\documentclass[11pt]{letter}
\usepackage[hidelinks]{hyperref}
\usepackage[green]{msuletterhead}
%\topmarginskip{-0.35in}
%\bottommarginskip{-1.5in}
%\leftmarginsize{1in}
%\rightmarginsize{1.25in}
%\sigskipbefore{0.4in}
%\sigskipafter{0in}
%\noLines
%\nolines
%\noHeadline
%\noheadline
%\signature{Gideon Bradburd, Graham Coop, and Peter Ralph}
%
%\namefrom{}
%\addrfrom{Gideon Bradburd \\ Dept. of Integrative Biology \\ Michigan State University \\ East Lansing, MI 48823 USA \\ 
%\href{mailto:bradburd@msu.edu}{bradburd@msu.edu}}
%\emailfrom{bradburd@msu.edu}
%
%\greetto{To the Editor(s) --}
%\closeline{Sincerely,}

\begin{document}
%\begin{newlfm}

Submission Cover Letter \hfill	\today\\
\emph{PLoS Biology}
\\\\\\
To the Editor(s),

Please find enclosed our manuscript,  
``Inferring Continuous and Discrete Population Genetic Structure Across Space,"
for consideration for publication in PLoS Biology.

In the manuscript, we present a novel statistical method 
for the simultaneous inference of continuous and discrete patterns of population structure.
The method estimates ancestry proportions for each 
sample from a set of discrete population layers, 
and, within each layer, estimates a rate at which relatedness decays with distance.
This model explicitly addresses the ``clines vs. clusters" problem in 
modeling population genetic variation by jointly accommodating 
continuous and discrete patterns of differentiation. 
We demonstrate the utility of this approach using simulations 
and two empirical applications. %, with datasets from poplars and black bears. 
We also release an associated R package, called \texttt{conStruct}, 
for ``\emph{con}tinuous \emph{struct}ure."

This work is motivated by the failure of existing model-based clustering methods, 
such as \texttt{STRUCTURE} (Pritchard \emph{et al.} 2000)
or \texttt{ADMIXTURE} (Alexander \emph{et al.} 2009), 
to model continuous patterns of geographical variaion (i.e., isolation by distance, or IBD).
Such geographical patterns are ubiquitous in nature,
but are described by these clustering methods using a possibly large number of discrete clusters.
These clusters are spurious, and can throw a 
wrench in the gears of efforts to describe and categorize natural variation, 
one of the first steps in any analysis of genetic data, 
and a principal mission of biology.  

Besides presenting a solution to this problem,
we use simulations and empirical data to demonstrate 
the ways in which nonspatial clustering approaches 
fail in the presence of IBD; 
and provide two different metrics for choosing an appropriate 
number of layers to describe the data.

As of this submission, \texttt{STRUCTURE} alone has been cited more than 20,000 times.
However, as the costs of sequencing continue to fall
and good geospatial information becomes more readily available,
the limitations we cite above will become more severe.
Our method has the potential to fill this gap.
% We therefore believe that our work represents an important and timely contribution to the field.

John Novembre (jnovembre@uchicago.edu), \\
Desislava Petkova (desislavka@gmail.com), \\
Jonathan Pritchard (pritch@stanford.edu),  \\
Sohini Ramachandran (sramachandran@brown.edu), \\
and Matthew Stephens (mstephens@uchicago.edu) \\
would be suitable reviewers. 

Sincerely,
Gideon Bradburd, 
Graham Coop, and 
Peter Ralph
%\end{newlfm}
\end{document}

%What is the scientific question you are addressing?
%What is the key finding that answers this question?
%What is the nature of the evidence you provide in support of your conclusion?
%What are the three most recently published articles that are relevant to this question?
%What significance do your results have for the field?
%What significance do your results have for the broader community (of biologists and/or the public)?
%What other novel findings do you present?
%Is there additional information that we should take into account?
